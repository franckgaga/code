% !TeX encoding = UTF-8
% !TeX spellcheck = en_US
\section{Introduction}

The control community, both in the academic and industrial sectors, has traditionally relied on MATLAB and its toolboxes for designing, simulating and implementing closed-loop systems. The ecosystem developed by The MathWorks\texttrademark\ is rich, mature, cohesive, and well documented, but their licensing policy can be unaffordable for smaller organizations.

Additionally, because it is a proprietary software, many functions are closed source, weakening the transparency and reproducibility for scientific research. \citep{matlabPythonJulia}

\citep{controlsystems_jl} \cite{jump_jl}


For large organizations, the  


The intense particle mixing and heat diffusivity of fluidized beds explains their popularity for pharmaceutical product drying. However, the gas flow through fluidized solids follows heterogeneous patterns with bypasses, generally perceptible as gas bubbles \citep{handbookFluidization}. Consequently, theories describing fluidized bed gas behavior are complex, leading to mathematical models hardly applicable for actual industrial developments \citep{reviewModelPhilippsen}. This is the case for process control design and implementation because computational burdens and tuning complexity hinder real-time applications \citep[e.g.][]{nmpcFBD}. Simplifying the \gls{FBD} dynamic model is thus desirable, at the very least for control purposes.

\citet{batchTsotas} and \citet{fbd2phFG} introduce two-phase batch \gls{FBD} models to describe gas flow patterns. Although more realistic than single-phase descriptions \citep{flowModelCompFB}, they generally lead to large differential and algebraic equation systems requiring complex solving workflow. 

A homogeneous single-phase dryer with nonuniform distributions for internal moisture in particles is described in \citet{batchCorn}. Distributed models cannot be applied in a straightforward manner to process control. Alternatively, the van Meel normalization \citep{vanMeelModel} or the characteristic drying curves concepts \citep{handbookDrying} enable a lumped description of sorption and internal diffusion kinetics, like the single-phase model of \citet{batchGavi}. Separate experimentation is however required to determine the curve function, the critical, and the equilibrium moisture content. Furthermore, theses parameters depend on inlet conditions and experimental equipment \citep{handbookDrying}. A homogeneous representation is described in \citet{batchThermo}, but simulations need outlet gas data as inputs. It must also be emphasized that an accurate sorption model is less crucial for applications dominated by the constant rate period (or first drying stage), which is common for pharmaceutical product. 

Other approaches include purely empirical or black-box models. However, incorporating the inlet gas condition effect in these strategies is difficult and ambiguous since impacts on empirical parameter values are not well-defined \citep{handbookDrying}. This feature is essential for process control applications since the gas feed rate and temperature are the main manipulated inputs during drying. Linear model identification and control are also not trivial since batch processes like \glspl{FBD} cannot be linearized around an operating point and exhibit irreversible reactions \citep{robustNMPC}. These features explain why their dynamics are generally classified as highly nonlinear and less suitable for classical control.

This paper presents a simple predictive controller for batch \glspl{FBD} based on a single-phase model. To this end, the two-phase equations of \citet{fbd2phFG} are reduced using simplifying assumptions and physical approximations. They are then used for \gls{NMPC} based on an internal model scheme, leading to an algorithm relatively easy to implement and tune. The model parameters are calibrated with a nonlinear grey-box identification algorithm on pharmaceutical pilot scale data. The validation compares simulations with the complete two-phase description on separate dataset. Implementing the control strategy on a simulated two-phase \gls{FBD} attests closed-loop performance and robustness. 