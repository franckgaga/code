% !TeX encoding = UTF-8
% !TeX spellcheck = en_US
\section{Introduction}

The control community, both in the academic and industrial sectors, has largely relied on MATLAB and its toolboxes for designing, simulating and implementing closed-loop systems. The ecosystem developed by MathWorks\texttrademark\ is rich, mature, cohesive, and well documented, but their licensing policy can be unaffordable for smaller organizations. Moreover, because it is a proprietary software, the source code of many functions is not available. This is an issue for scientific research, where reproducibility and transparency is a key aspect. Lastly, like any interpreted programming languages, its performance can be suboptimal for computationally intensive tasks, especially for time-critical applications like real time optimization and model predictive control \citep{matlabPythonJulia}.

Julia is relatively new programming language specialized for scientific computing. The just-ahead-of-time compiler allows reaching performance comparable to C and Fortran, while exhibiting a modern and expressive syntax like MATLAB and Python. The built-in read-eval-print-loop (REPL) allows to interactively test code and inspect variables, mimicking the development workflow of an interpreted language. Moreover, the language is free and open-source and it can be used for commercial purposes without any licensing fees. The ecosystem is still young, but a control toolbox, a system identification and an optimization packages are already available and complete for the most parts \citep{controlsystems_jl, jump_jl}. There is however no open-source model predictive control package available yet in the official registry of Julia, which is the main motivation for this project.

This paper presents a model predictive control package for julia. A simple and creative framework for the development of advanced process controller. The focus is on modern mpc that rely on a state observer. Traditional approach like internal model structure is also available. The \textttsmall{JuMP.jl} interface allows changing the solver with a one-liner. Automatic differentiation (AD) compute the gradient and jacobians of models (EKF), and objective and constrain function of the nonlinear model predictive controller. Both soft and hard constraints are supported. Linear representation are automatically augmented with an adequate stochastic model of the disturbances (can be customized). The classical Kalman-type states estimator are all available as state feedback strategies. The analogue of predictive control for estimation, the moving horizon estimation, will be also available soon after this publication. This observer allows adding constraints on the estimates. 

\section{Materials and methods}

The code of this package is separated into three abstract types:
\begin{description}
    \item[SimModel] (2 subtypes) -- Discrete state-space models of the plant, including linear and nonlinear representations. They serve as wrappers to construct \textbf{StateEstimator}s and \textbf{PredictiveController}s, and also as plant simulators.
    \item[StateEstimator] (6 subtypes) -- Deterministic and stochastic observers used as the feedback for the \textbf{PredictiveController}s
    \item[PredictiveController] (3 subtypes) -- Linear and nonlinear MPC are available. An explicit controller based on matrix algebra is also possible for linear description without constraint.
\end{description}

\section{Case studies}

The example considers a continuisly stirred tank reactor (CSTR) with a hot and cold water intakes. The linear model, with the manipulated inputs $\mathbf{u}=\begin{smallmatrix}[
u_c & u_h]'\end{smallmatrix}$ and the liquid level and tempertaure as measured outputs $\mathbf{y}=\begin{smallmatrix}[y_L & y_T]'\end{smallmatrix}$ is constructed with:

\begin{minted}{julia}
using ModelPredictiveControl, ControlSystemsBase
sys = [ tf(1.90, [18, 1]) tf(1.90, [18, 1]);
        tf(-0.74,[8, 1])  tf(0.74, [8, 1]) ]
Ts = 4.0
model = setop!(LinModel(sys, Ts), yop=[50, 30], uop=[20, 20])
\end{minted}
\vspace{-26pt}
\begin{minted}{julia-repl}
Discrete-time linear model with a sample time 
Ts = 4.0 s and:
 2 manipulated inputs u
 2 states x
 2 outputs y
 0 measured disturbances d
\end{minted}

The figure 1 depicts the instrumentation installed on the plant.

Ut dolore reprehenderit sint cillum do occaecat et pariatur duis. Culpa veniam esse aute velit ex. Aliquip est nisi ullamco fugiat excepteur incididunt nulla proident culpa sint nostrud id id. Dolor non culpa mollit nostrud ut dolore velit qui est aliqua.

The objective is to control both the water temperature and level and enforcing a 45\% or 
more liquid level:

\begin{minted}{julia}
ŷmin=[45, -Inf]
mpc = setconstraint!(LinMPC(model, Hp=15); ŷmin)
\end{minted}
\vspace{-25pt}
\begin{minted}{julia-repl}
LinMPC controller with a sample time Ts = 4.0 s,
OSQP optimizer, SteadyKalmanFilter estimator and:
 15 prediction steps Hp
  2 control steps Hc
  2 manipulated inputs u (0 integrating states)
  4 states x̂
  2 measured outputs ym (2 integrating states)
  0 unmeasured outputs yu
  0 measured disturbances d
\end{minted}

