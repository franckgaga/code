% !TeX encoding = UTF-8
% !TeX spellcheck = en_US
\section{Case studies}

\subsection{Continously Stirred Tank Reactor}

\subsubsection{Linear Model}

The example considers a continuously stirred tank reactor with a hot and cold water intakes. The linear model, with the manipulated inputs $\mathbf{u}=\begin{smallmatrix}[
u_c & u_h]'\end{smallmatrix}$ and the liquid level and tempertaure as measured outputs $\mathbf{y}=\begin{smallmatrix}[y_L & y_T]'\end{smallmatrix}$ is constructed with:

\begin{minted}{julia}
using ModelPredictiveControl, ControlSystemsBase
sys = [ tf(1.90, [18, 1]) tf(1.90, [18, 1]);
        tf(-0.74,[8, 1])  tf(0.74, [8, 1]) ]
Ts = 2.0; uop = [20, 20]; yop = [50, 30]
model = setop!(LinModel(sys, Ts); uop, yop)
\end{minted}
\vspace{-26pt}
\begin{minted}{julia-repl}
Discrete-time linear model with a sample time 
Ts = 2.0 s and:
 2 manipulated inputs u
 2 states x
 2 outputs y
 0 measured disturbances d
\end{minted}

The figure 1 depicts the instrumentation installed on the plant. The \texttt{model} object will be used for two purposes : to construct our controller, and as a plant simulator to test the design.

\subsubsection{Linear Model Predictive Controller}

The objective is to control both the water temperature and level while constraining the level above 45:

\begin{minted}{julia}
nint_u = [1, 1]; ymin = [45, -Inf]
mpc = setconstraint!(LinMPC(model; nint_u); ymin)
\end{minted}
\vspace{-25pt}
\begin{minted}{julia-repl}
LinMPC controller with a sample time Ts = 2.0 s,
OSQP optimizer, SteadyKalmanFilter estimator and:
 15 prediction steps Hp
  2 control steps Hc
  2 manipulated inputs u (2 integrating states)
  4 states x̂
  2 measured outputs ym (0 integrating states)
  0 unmeasured outputs yu
  0 measured disturbances d
\end{minted}

By default, \texttt{LinMPC} controllers use \texttt{OSQP.jl} to solve the problem, soft constraints on output predictions $\mathbf{\hat y}$ to ensure feasibility, and a SteadyKalmanFilter to estimate the plant states\footnote{We could have use an InternalModel structure, to avoid state observer, with \texttt{mpc = LinMPC(InternalModel(model), Hp=15, Hc=2, Mwt=[1, 1], Nwt=[0.1, 0.1])}. It was tested on the example of this page and it gives similar results.}. An attentive reader will also notice that the Kalman filter estimates two additional states compared to the plant model. These are the integrating states for the unmeasured plant disturbances, added at the model inputs here.  

Before closing the loop, 

we initialize the estimates with the actual plant inputs and measurements to ensure a bumpless transfer. Since model simulates our plant here, its output will initialize the states. \texttt{LinModel} objects are callable for this purpose. 

We can then close the loop and test \texttt{mpc} performance on the simulator by imposing step changes on output setpoints $\mathbf{r_y}$ and on a load disturbance $u_l$:

\begin{minted}{julia}
function test_mpc(mpc, plant)
    plant.x[:] .= 0; y = plant() # or evaloutput(plant)
    initstate!(mpc, plant.uop, y)
    N = 75; ry = [50, 30]; ul = 0
    U_data, Y_data = zeros(plant.nu,N), zeros(plant.ny,N)
    Ry_data = zeros(plant.ny,N)
    for i = 1:N
        i == 26 && (ry = [48, 35])
        i == 51 && (ul = -10)
        y = plant() 
        u = mpc(ry) # or moveinput!(mpc, ry)
        U_data[:,i], Y_data[:,i], Ry_data[:,i] = u, y, ry
        updatestate!(mpc, u, y) # update mpc estimate
        updatestate!(plant, u+[0,ul])
    end
    return U_data, Y_data, Ry_data
end
U_data, Y_data, Ry_data = test_mpc(mpc, model)
\end{minted}

Updating the internal states of \texttt{mpc} prepares the object for the \emph{next} time step. That is why the call is done at the end of the \texttt{for} loop. The same logic applies for \texttt{model}. Lastly, we plot the closed-loop test with \texttt{Plots.jl}:

\begin{minted}{julia}
res = SimResult(mpc, U_data, Y_data; Ry_data)
using Plots; plot(res)
\end{minted}

\begin{figure}
    \centering
    \includegraphics[width=\columnwidth]{fig/plot1_LinMPC.pdf}
    \caption{CSTR closed-loop simulation}
    \label{fig:plot1_LinMPC}
\end{figure}

\cref{fig:plot1_LinMPC} shows that the controller violates the constraints around 110 s because of the disturbance. Adding feedforward compensation can improve the performance.

\subsubsection{Feedforward Compensation}

Suppose that the load disturbance $u_l$ of the last section is in fact caused by a separate hot water pipe that discharges into the tank. Measuring this flow rate allows us to incorporate feedforward compensation:

\begin{minted}{julia}
model_d = LinModel([sys sys[1:2, 2]], Ts; i_d=[3])
model_d = setop!(model_d; uop, yop, dop=[20])
\end{minted}

The simulation needs an MPC based on \texttt{model\_ff} and a new test function that employs the measured disturbance:

\begin{minted}{julia}
mpc = setconstraint!(LinMPC(model_d; nint_u); ymin)
function test_mpc_ff(mpc, plant)
    plant.x[:] .= 0; y = plant(); d = [20]
    initstate!(mpc, plant.uop, y, d)
    N = 75; ry = [50, 30]; ul = 0
    U_data, Y_data = zeros(plant.nu,N), zeros(plant.ny,N)
    Ry_data = zeros(plant.ny,N)
    for i = 1:N
        i == 26 && (ry = [48, 35])
        i == 51 && (ul = -10)
        y = plant(); d = [20+ul]
        u = mpc(ry, d)
        U_data[:,i], Y_data[:,i], Ry_data[:,i] = u, y, ry
        updatestate!(mpc, u, y, d) 
        updatestate!(plant, u+[0,ul])
    end
    return U_data, Y_data, Ry_data
end
U_data, Y_data, Ry_data = test_mpc_ff(mpc, model)
plot(SimResult(mpc, U_data, Y_data; Ry_data))
\end{minted}

\cref{fig:plot2_LinMPC} shows that the compensation handles the disturbance without violating the constraint.

\begin{figure}
    \centering
    \includegraphics[width=\columnwidth]{fig/plot2_LinMPC.pdf}
    \caption{CSTR closed-loop simulation with feedforward}
    \label{fig:plot2_LinMPC}
\end{figure}

\subsection{Inverted Pendulum}



\section{Results and Discussion}
\label{sec.results_simple}

Laboris exercitation sunt adipisicing laboris nulla ad occaecat irure magna voluptate. Veniam eu labore Lorem anim nostrud minim Lorem fugiat nulla. Non anim veniam laborum deserunt adipisicing cillum aliquip deserunt pariatur consectetur consequat irure. Dolor aute incididunt quis magna. Dolor nostrud exercitation duis sunt nulla adipisicing enim laborum ex ut esse ad. Incididunt exercitation velit ea elit ut laboris eu amet.

Minim magna culpa ipsum do proident cillum sit. Officia ad et consequat do magna nostrud irure nostrud. Ipsum sunt adipisicing Lorem sunt tempor ex veniam consequat veniam ipsum. Aliquip voluptate labore mollit duis id mollit nulla id ipsum. Velit esse nisi qui pariatur ad nostrud.

Aute minim anim culpa minim aliquip sint laboris sunt. Ea ex tempor irure officia laborum est aliquip nostrud quis do. Lorem cupidatat mollit do officia aute in est cillum cillum consectetur laborum pariatur. Id tempor minim eiusmod ex laboris. Aute aliquip cupidatat nostrud aute dolor mollit ex commodo do excepteur dolor irure. Ut veniam amet id excepteur excepteur velit laboris aliqua sit occaecat irure dolor labore.

Proident ad minim magna minim excepteur reprehenderit elit id ipsum amet velit sunt aute aute. Consectetur excepteur velit quis aliqua sint laborum veniam minim proident. Veniam proident laboris aute deserunt do do amet qui pariatur occaecat cillum. Minim magna in et laborum ipsum deserunt pariatur cupidatat ipsum veniam aliqua elit. Aliqua id proident fugiat nisi esse eu nulla sint qui reprehenderit aliqua sunt. Cillum duis anim sit velit dolore.

\begin{table}[tb]
	\caption{Simplified model parameters for batch fluidized bed drying}
	\label{tab.simple_params}
	\centering
	% !TeX spellcheck = en_US

\begin{tabular}{l l l l}
	
\toprule %=======================================================================

parameter 			& value 	   		& \shortstack[l]{standard\\deviation} & units  \\

\midrule %-----------------------------------------------------------------
					
\glsub{chi}{0}    	& \num{0.15e-2}		& -- 					& --		\\

%--------------------------------------------------------------------------

\rowcolor{blue!10}
\gls{a1}			& \num{9.26e-5}		& \num{0.04e-5}		%
										& \si{\per\meter\cubed} \\
\rowcolor{blue!10}
\gls{b1}			& \num{4.54e-2}		& \num{0.05e-2}		%
										& \si{\degreeCelsius\per\meter\cubed} \\
\rowcolor{blue!10}					
\gls{b2}			& \num{2.56e-05}	& \num{0.02e-05}	%
										& \si{\per\meter\cubed}	 \\

%--------------------------------------------------------------------------

\rowcolor{blue!10}
\gls{nu}			& \num{3.48}		&  \num{0.12}		& -- \\		
\rowcolor{blue!10}
\glsub{chi}{pc}		& \num{2.33e-2}	 	&  \num{0.06e-2}	& -- \\
				
%-------------------------------------------------------------------------

\gls{alpha}   	   	& \num{4.90e-3} 	& --			& --  \\
\gls{beta}			& \num{5.70e-2}		& -- 			& \si{\per\degreeCelsius}\\

%----------------------------------------------------------------------------

\gls{tau}			& \num{86.7}		&--				& \si{\second}  \\

\bottomrule %====================================================================
	
\end{tabular}

\end{table}

Occaecat Lorem commodo in minim consectetur voluptate nostrud enim tempor incididunt laborum occaecat. Aliqua non et nulla voluptate amet sunt. Laboris nisi consequat cupidatat est consequat deserunt ipsum.
